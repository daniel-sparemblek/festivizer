\chapter{Architecture and System Design}

System architecture can be divided into three sub-systems :
\begin{itemize}
	\item 	 Web server
	\item 	 Android application
	\item 	 Database
\end{itemize}

\textbf{Android application}
It allows user to view and interact with the specific components of the graphic interface. In Android, graphical components are written in the XML (Extensible Markup Language) which is the language that can be understood by human and by the machine. On interaction with the GUI elements certain parts of the Android code are executed, mostly the ones that send HTTP requests to the Web Server. To send HTTP requests we are using Android’s Volley library.

\textbf{Web Server}
Web Server is the one who responds to the HTTP requests that are sent by Android application. Primary task of the web server is communication with the client (Android application) and fetching data from the database. Required data is then sent back to the client that displays data on the GUI. Response can be sent in the JSON (JavaScript Object Notation) format or as the plain String. Web server is also the one responsible for authentication and authorization of the user, server checks if the user has permission to access the certain data. If user does not have the required permission, Server will send Error response back to the client whose task is to handle the error properly.

Programming language in which we have coded the Android application is Java, GUI components are written in the XML language and server-side code is written in Python. IDE (Integrated development environment) we are using is Android Studio for the client-side code and IDLE for the server-side code.

Design pattern we are using is MVC(Model-View-Controller). It is a common architectural pattern which is used to design and create interfaces and the structure of an application. This pattern divides the application into three parts that are dependent and connected to each other. These designs are used to distinguish the presentation of data from the way the data is accepted from the user to the data that is being shown. 
MVC design pattern consists of: 

\begin{itemize}
	\item 	 Model
	\item 	 View
	\item 	 Controller		
\end{itemize}

\textbf{Model}
The Model component corresponds to all the data-related logic that the user works with. This can represent either the data that is being transferred between the View and Controller components or any other business logic-related data. For example, a Customer object will retrieve the customer information from the database, manipulate it and update it data back to the database or use it to render data. 
In Android application Model is usually represented as independent Java class that contains data about e.g. Customer. That data is the one that is kept in the data base and displayed in the View component. 

\textbf{View}
The View component is used for all the UI logic of the application. For example, the Customer view will include all the UI components such as text boxes, dropdowns, etc. that the final user interacts with. 
In our application an example of View would be Activity for register, user interacts with the View, enters the necessary information that are then transferred to the Model and finally to the database via Controller.

\textbf{Controller}
Controllers act as an interface between Model and View components to process all the business logic and incoming requests, manipulate data using the Model component and interact with the Views to render the final output. For example, the Customer controller will handle all the interactions and inputs from the Customer View and update the database using the Customer Model. The same controller will be used to view the Customer data.
\pagebreak


\section{Database}

Database used for this project is a relation based database made in SQLite. Relation is usually referred to as a table that has tuples. Tuple is an object that represents an information. Purpose of the database is easy and fast data manipulation, including saving, deleting, updating and sending data to the server. Database has relations:
\begin{itemize}
	\item 	 Users
	\item 	 Festivals
	\item 	 Events
	\item 	 Specializations
	\item 	 WorkerSpecializations
	\item 	 Auctions
	\item 	 Application
	\item 	 Jobs
	\item 	 JobSpecializations
	\item 	 FestivalOrganizers
	
\end{itemize}
\pagebreak
\subsection{Tables details}


\textbf{User}  has entities for every user of the app. It has all needed personal information about the user and his role.

\begin{longtabu} to \textwidth {|X[6, l]|X[6, l]|X[20, l]|}
	
	\hline \multicolumn{3}{|c|}{\textbf{User}}	 \\[3pt] \hline
	\endfirsthead
	
	\hline 
	\endlastfoot
	
	\cellcolor{LightGreen}user\_id & INT	&  	User identification number 	\\ \hline
	username	& VARCHAR &  Unique username 	\\ \hline 
	password & VARCHAR & User account password  \\ \hline 
	firstname & VARCHAR	&  Users first name	\\ \hline 
	lastname & VARCHAR	&  Users last name	\\ \hline 
	picture & VARCHAR	&  Profile picture string	\\ \hline 
	phone & VARCHAR	&  Users phone number	\\ \hline 
	email & VARCHAR	&  Users email address	\\ \hline 
	permission & INT	&  Level of permission that user has	\\ \hline
	is\_pending & INT	&  Acceptance status \\ \hline
	
\end{longtabu}

\textbf{Festival}  contains all needed information about the festival.

\begin{longtabu} to \textwidth {|X[6, l]|X[6, l]|X[20, l]|}
	
	\hline \multicolumn{3}{|c|}{\textbf{Festival}}	 \\[3pt] \hline
	\endfirsthead
	
	\hline 
	\endlastfoot
	
	\cellcolor{LightGreen}festival\_id & INT	&  	Festival identification number 	\\ \hline
	\cellcolor{LightBlue}leader\_id	& INT &  ID of the user who created the festival 	\\ \hline 
	name & VARCHAR & Festival name  \\ \hline 
	desc & VARCHAR	&  Short festival description, can be empty	\\ \hline 
	logo & VARCHAR	&  Festivals logo string	\\ \hline 
	start\_time & VARCHAR	&  Festivals start time	\\ \hline 
	end\_time & VARCHAR	&  Festivals end time	\\ \hline 
	status & BOOLEAN	&  True if festival is active, False if it's inactive	\\ \hline 
	
\end{longtabu}

\textbf{Event}  contains information about the event of the festival.

\begin{longtabu} to \textwidth {|X[6, l]|X[6, l]|X[20, l]|}
	
	\hline \multicolumn{3}{|c|}{\textbf{Event}}	 \\[3pt] \hline
	\endfirsthead
	
	\hline 
	\endlastfoot
	
	\cellcolor{LightGreen}event\_id & INT	&  	Event identification number 	\\ \hline
	\cellcolor{LightBlue}festival\_id	& INT &  ID of the festival that event belongs to 	\\ \hline 
	\cellcolor{LightBlue}organizer\_id 	& INT &  ID of the user who created the event  	\\ \hline 
	name & VARCHAR & Event name  \\ \hline 
	desc & VARCHAR	&  Short event description, can be empty	\\ \hline 
	location & VARCHAR	&  Events location	\\ \hline 
	start\_Time & TIMESTAMP	&  Event start time	\\ \hline 
	end\_Time & TIMESTAMP	&  Event end time  \\ \hline 
	
\end{longtabu}


\textbf{Specialization}  contains all different specializations and their names.

\begin{longtabu} to \textwidth {|X[7, l]|X[6, l]|X[19, l]|}
	
	\hline \multicolumn{3}{|c|}{\textbf{Specialization}}	 \\[3pt] \hline
	\endfirsthead
	
	\hline 
	\endlastfoot
	
	\cellcolor{LightBlue}specialization\_id & INT	&  	Specializations identification number 	\\ \hline
	name & VARCHAR & Name of the specialization \\ \hline
	
	
\end{longtabu}

\textbf{WorkerSpec}  contains specifications about the user who is a worker and his specializations.

\begin{longtabu} to \textwidth {|X[7, l]|X[6, l]|X[19, l]|}
	
	\hline \multicolumn{3}{|c|}{\textbf{WorkerSpec}}	 \\[3pt] \hline
	\endfirsthead
	
	\hline 
	\endlastfoot
	\cellcolor{LightGreen}worker\_specializations\_id & INT	&  	Unique tuple identification number 	\\ \hline
	\cellcolor{LightBlue}worker\_id & INT	&  	Workers identification number 	\\ \hline
	\cellcolor{LightBlue}specialization\_id & INT	&  	Specializations identification number 	\\ \hline
	
	
\end{longtabu}

\textbf{Auction}  contains information about auctions.

\begin{longtabu} to \textwidth {|X[6, l]|X[6, l]|X[20, l]|}
	
	\hline \multicolumn{3}{|c|}{\textbf{Auction}}	 \\[3pt] \hline
	\endfirsthead
	
	\hline 
	\endlastfoot
	
	\cellcolor{LightGreen}auction\_id & INT	&  	Auction identification number 	\\ \hline
	job\_id & INT & Identification number of the job that auction is for \\ \hline
	start\_Time & TIMESTAMP	&  Auction start time	\\ \hline 
	end\_Time & TIMESTAMP	&  Auction end time  \\ \hline 
	
\end{longtabu}


\textbf{Application}  contains information about applications for auctions.

\begin{longtabu} to \textwidth {|X[8, l]|X[6, l]|X[18, l]|}
	
	\hline \multicolumn{3}{|c|}{\textbf{Application}}	 \\[3pt] \hline
	\endfirsthead
	
	\hline \multicolumn{3}{|c|}{\textbf{Application}}	 \\[3pt] \hline
	\endhead
	
	\hline 
	\endlastfoot
	
	\cellcolor{LightGreen}application\_id & INT	&  	Application identification number 	\\ \hline
	\cellcolor{LightBlue}auction\_id & INT	&  	Auction identification number that worker applies for 	\\ \hline
	\cellcolor{LightBlue}worker\_id & INT	&  	Workers identification number 	\\ \hline
	price & FLOAT & Offered pay for the job \\ \hline
	comment & VARCHAR & Additional comment for application, can be empty \\ \hline
	duration & INT & Time needed to complete the job, in days \\ \hline
	people\_number & INT & Number of people that will be doing the job \\ \hline
	status & INT & Shows if application is accepted, declined or pending by the organizer \\ \hline
	
	
\end{longtabu}


\textbf{Job}  contains information about jobs that organizer created.

\begin{longtabu} to \textwidth {|X[6, l]|X[6, l]|X[20, l]|}
	
	\hline \multicolumn{3}{|c|}{\textbf{Job}}	 \\[3pt] \hline
	\endfirsthead
	
	\hline \multicolumn{3}{|c|}{\textbf{Job}}	 \\[3pt] \hline
	\endhead
	
	\hline 
	\endlastfoot
	
	\cellcolor{LightGreen}job\_id & INT	&  	Job identification number 	\\ \hline
	\cellcolor{LightBlue}event\_id & INT	&  	Events identification number that job is for\\ \hline
	\cellcolor{LightBlue}worker\_id & INT	&  	Workers identification number that does the job\\ \hline
	name & VARCHAR & Name of the job \\ \hline
	description & VARCHAR & Description of the job \\ \hline
	start\_Time & DATETIME	&  Jobs start time	\\ \hline 
	is\_Completed & BOOLEAN	&  True if job is finished, false if it's still active \\ \hline 
	comment & VARCHAR & Comment about the completed job that organizer puts in \\ \hline
	order\_number & INT & Order number of that job on certain event in comparison to other jobs \\ \hline
	
	
\end{longtabu}


\textbf{JobSpec}  contains specializations that are needed for the job.

\begin{longtabu} to \textwidth {|X[7, l]|X[6, l]|X[19, l]|}
	
	\hline \multicolumn{3}{|c|}{\textbf{JobSpec}}	 \\[3pt] \hline
	\endfirsthead
	
	\hline \multicolumn{3}{|c|}{\textbf{JobSpec}}	 \\[3pt] \hline
	\endhead
	
	\hline 
	\endlastfoot
	
	\cellcolor{LightGreen}job\_id & INT	&  	Job identification number 	\\ \hline
	\cellcolor{LightBlue}specialization\_id & INT	&  	Specializations identification number 	\\ \hline
	
	
\end{longtabu}


\textbf{FestivalOrganizers}  contains information about which user, that is also an organizer, applied for which festival.

\begin{longtabu} to \textwidth {|X[6, l]|X[6, l]|X[20, l]|}
	
	\hline \multicolumn{3}{|c|}{\textbf{FestivalOrganizers}}	 \\[3pt] \hline
	\endfirsthead
	
	\hline 
	\endlastfoot
	
	\cellcolor{LightGreen}festival\_id & INT	&  	Festival identification number 	\\ \hline
	\cellcolor{LightGreen}organizer\_id & INT	&  	Organizers identification number 	\\ \hline
	status & INT	&  \-1 when organizer is waiting on leader, 0 when rejected and 1 when accepted \\ \hline 
	
	
\end{longtabu}
\pagebreak
\subsection{Database diagrams}


\begin{figure}[H]
	\includegraphics[width=\linewidth]{diagrams/db_er_diag.png}
	\centering
	\caption{E-R Diagram}
	\label{fig:er_diag}
\end{figure}

\begin{figure}[H]
	\includegraphics[width=\linewidth]{diagrams/er_diagram.png}
	\centering
	\caption{Database Diagram}
	\label{fig:normal_diag}
\end{figure}

\eject

\pagebreak
\section{Class Diagram}

Generally, classes can be split into 6 parts:
\begin{packed_enum}
	\item Activity classes
	\item Controller classes
	\item API Classes
	\item Fragment Classes
	\item Model Classes
	\item Adapter Classes
\end{packed_enum}

Controller classes are used for communication between the server and the mobile "front-end" application. They handle all the logic and data exchange. They are the ones who actually use all the other Classes.

\begin{figure}[H]
	\includegraphics[width=\linewidth]{diagrams/Controllers Class Diagram_part1.png}
	\caption{Controller Class Diagram 1}
	\label{fig:controller_class_diag_pt1}
\end{figure}

\begin{figure}[H]
	\includegraphics[width=\linewidth]{diagrams/Controllers Class Diagram_part2.png}
	\caption{Controller Class Diagram 2}
	\label{fig:controller_class_diag_pt2}
\end{figure}

\begin{figure}[H]
	\includegraphics[width=\linewidth]{diagrams/Controllers Class Diagram_part3.png}
	\caption{Controller Class Diagram 3}
	\label{fig:controller_class_diag_pt3}
\end{figure}

\begin{figure}[H]
	\includegraphics[width=\linewidth]{diagrams/Controllers Class Diagram_part4.png}
	\caption{Controller Class Diagram 4}
	\label{fig:controller_class_diag_pt4}
\end{figure}

\begin{figure}[H]
	\includegraphics[width=\linewidth]{diagrams/Controllers Class Diagram_part5.png}
	\caption{Controller Class Diagram 5}
	\label{fig:controller_class_diag_pt5}
\end{figure}

\begin{figure}[H]
	\includegraphics[width=\linewidth]{diagrams/Controllers Class Diagram_part6.png}
	\caption{Controller Class Diagram 6}
	\label{fig:controller_class_diag_pt6}
\end{figure}

\begin{figure}[H]
	\includegraphics[width=\linewidth]{diagrams/Controllers Class Diagram_part7.png}
	\caption{Controller Class Diagram 7}
	\label{fig:controller_class_diag_pt7}
\end{figure}

The Model Classes are used to hold all the data. Therefore, theya re the intermediary classes used for storing data during data exchanges with the server adn the database. Data is pulled into them, as well as pushed from them into the database. Basically, they represent data storage.

\begin{figure}[H]
	\includegraphics[width=\linewidth]{diagrams/Models Class Diagram_1.png}
	\caption{Models Class Diagram 1}
	\label{fig:models_class_diag_1}
\end{figure}

\begin{figure}[H]
	\includegraphics[width=\linewidth]{diagrams/Models Class Diagram_2.png}
	\caption{Models Class Diagram 2}
	\label{fig:models_class_diag_2}
\end{figure}

\begin{figure}[H]
	\includegraphics[width=\linewidth]{diagrams/Models Class Diagram_3.png}
	\caption{Models Class Diagram 3}
	\label{fig:models_class_diag_3}
\end{figure}

\begin{figure}[H]
	\includegraphics[width=\linewidth]{diagrams/Models Class Diagram_4.png}
	\caption{Models Class Diagram 4}
	\label{fig:models_class_diag_4}
\end{figure}

\begin{figure}[H]
	\includegraphics[width=\linewidth]{diagrams/Models Class Diagram_5.png}
	\caption{Models Class Diagram 5}
	\label{fig:models_class_diag_5}
\end{figure}

\begin{figure}[H]
	\includegraphics[width=\linewidth]{diagrams/Models Class Diagram_6.png}
	\caption{Models Class Diagram 6}
	\label{fig:models_class_diag_6}
\end{figure}

API Classes are used for defining the mobile application - server(-> database)communication interface, and their diagrams will not be displayed here due to them being just interfaces.

Activity classes are basically the "front-end" classes of the application. They in cooperation with .XML files represent the static and the dynamic display of pages to the User.

XML classes provide static look, while these Activity classes provide dynamic look, as well as various UI functionalities, transitions, ...

\begin{figure}[H]
	\includegraphics[width=\linewidth]{diagrams/Activities Class Diagram_1.png}
	\caption{Activities Class Diagram 1}
	\label{fig:activities_class_diag_1}
\end{figure}

\begin{figure}[H]
	\includegraphics[width=\linewidth]{diagrams/Activities Class Diagram_1.png}
	\caption{Activities Class Diagram 2}
	\label{fig:activities_class_diag_2}
\end{figure}

\begin{figure}[H]
	\includegraphics[width=\linewidth]{diagrams/Activities Class Diagram_1.png}
	\caption{Activities Class Diagram 3}
	\label{fig:activities_class_diag_3}
\end{figure}

Adapter are basically used to hold Controllers, lists, and arrays of data. They are merely a utility class that is necessary for Android Studio to handle these multi-object constructs properly.

\begin{figure}[H]
	\includegraphics[width=\linewidth]{diagrams/Adapters Class Diagram.png}
	\caption{Adapters Class Diagram}
	\label{fig:adapters_class_diag}
\end{figure}

Finally, Fragments are used for tabbed views inside a single Activity -> something like a Sub-Activity, but with an easy and seamless transition.

\begin{figure}[H]
	\includegraphics[width=\linewidth]{diagrams/Fragments Class Diagram.png}
	\caption{Fragments Class Diagram}
	\label{fig:fragments_class_diag}
\end{figure}

\section{State diagram}

This diagram displays the abilities and functions of a Leader shown in a state-machine form. The Leader's functionalities and states revolve around Festivals.

To begin with, Leaders can create Festivals. They need to fill out the corresponding form and info. Once a Festival is made, it is added to the list of Festivals.

The Leaders can then view this list of all the Festivals. They are divided into 3 categories:
\begin{packed_enum}
	\item Active - festivals that are currently being carried out
	\item Pending - festivals that are yet to start
	\item Completed - festivals that have concluded
\end{packed_enum}

Second of all, on this Screen they can tap on a Festival to view its details. Here further Festival management is possible:
\begin{packed_enum}
	\item Events - View the list of Events
	\item Add new Event - Add new Events to this Festival
	\item Approve Organizers - Approve an Organizer to the selected Festival
\end{packed_enum}

Furthermore, Leaders can view a list of Job Applications. There all the Job Applications and their details are displayed in a list.

Finally, Leaders have also got the generic functionalities of logging out, printing a Festival Pass and searching Users.

\begin{figure}[H]
	\includegraphics[width=\linewidth]{diagrams/Leader State Diag.png}
	\caption{Leader State Diagram}
	\label{fig:leader_state_diag}
\end{figure}
\eject

\section{Activity diagram}

Two activity diagrams follow. The first one depicts the registration process, while the second one depicts a more complex system interaction.

\begin{figure}[H]
	\includegraphics[width=\linewidth]{diagrams/Activity Diagram - Register.png}
	\caption{Register Activity Diagram}
	\label{fig:register_activity_diag}
\end{figure}

In the second diagram we have a multiple-User interaction. In the beginning, the Organizer logs in and applies to one of the Festivals. Therefore he must select a Festival.

Upon Festival selection he needs to be verified by that Festival's Organizer. When that is done, Organizer can select an Event from a Festival.

The Leader then creates a Job and opens it up to Workers, and finally, one Worker applies to it.

This diagram shows the interaction between logging in, Leaders, Organizers, and Workers. This interaction is shown in a relative time-domain.

\begin{figure}[H]
	\includegraphics[width=\linewidth]{diagrams/Interaction_Activity_Diag_1.png}
	\caption{Interaction Activity Diagram 1}
	\label{fig:interaction_activity_diag_1}
\end{figure}

\begin{figure}[H]
	\includegraphics[width=\linewidth]{diagrams/Interaction_Activity_Diag_2.png}
	\caption{Interaction Activity Diagram 2}
	\label{fig:register_activity_diag_2}
\end{figure}

\eject

\section{Component diagram}

The application can so be divided into 2 separate parts - the front-end android application part, and the back-end server part. Even though a mobile application, this division much resembles the usual architecture of web applications.

To begin with, the back-end is the part hosted on the pythonanywhere cloud. It features the python back-end code and the SQLite database. The python script called 'Run.py' is the one which starts the Marshmallow, JWT and SQLAlchemy Python components. From there, other components - 'Models.py' and 'Resources.py' can continue to work. The 'Models.py' scripts basically contains Database models in Python class form, while the 'Resources.py' script handles server requests and responses.

As is visible, the Android application and the server need 2 interfaces - one for requests, and the other one for responses. The Android application is actually split into multiple parts: Controllers, API classes, Activities, and Models. This is due to the implementation of the MVC architecture, where Activities are the Android equivalent of Views. API Classes are the ones that provide the JSON GET, POST, and PUT interface methods for communication with the Server. All the logic is handled in the Controllers. They are the ones that call API methods, fetch, and send data, as well as update the Models and the Activities.

Activities control and manifest the design - they provide dynamic view experience to the user. Design layouts are actually .xml files that were made in Android Studio. Finally, Models are made by controllers, and they resemble the back-end Python models. They are class manifestation of database tables - and are used for holding dynamic data. Data is fetched into them, as well as pulled from them when updating server info.

The adapter class is used for holding multiple instances(lists) of some Models, and the utility class contains many subclasses that were hard to categorise into one of the other classes. Search is used for searching users --> found users are displayed using the defaultUser activity/view class.

\begin{figure}[H]
	\includegraphics[width=\linewidth]{diagrams/Component Diagram0.png}
	\caption{Component Diagram}
	\label{fig:component_diag}
\end{figure}

\eject