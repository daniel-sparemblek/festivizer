\chapter{Conclusion and Outline of Planned Future Work}
		
		Task of our group was to develop mobile phone application that allows organization of the festivals. After working hard for 13 weeks, we managed to develop alpha version of the application that satisfies given task. Work on this group project can be partitioned into two parts: (1) Organization of development, (2) Bare development and documentation.
		
		Organization of development includes gathering team members, assigning a task to our group by our teaching assistant, and work on documentation requirements. Documentation requirements that we set for ourselves were slightly changed during development, due to the progress and challenges in development.
		
		Bare development and documentation includes, as it says, development of application that satisfies given task (that we called 'Festivizer' and made some really cool custom logo) and parallel to development we wrote documentation. Some of our team mates were new in field of programming Android applications so they had to gain new skills. Learning that skills was left on us/them who didn't know that. But as things get going, that was no longer a problem. We started a little bit slow, but we managed to gain high development speed thanks to our team work. At beginning experienced team mates (we had couple of them) helped rest of us to get into new filed. As we were developing application that includes front-end (Android application) and back-end (server and database) we wrote documentation. Although, mostly we firstly completed development of certain part of application and than we wrote documentation for that part. And that is one thing of all that we would changed if would develop some other application together. We would firstly well define application requirements in documentation and after that we would start to develop using some of iterative development options. With that approach we would be more productive, and even if we realize that we misjudged some part of development plan, we could slightly changed requirements and continue to move forward - quickly and without stopping.
		
		Team communication was firstly done using WhatsApp, but soon we realized that it is not very practical so we moved to better platform for teamwork - Slack. In Slack we had different chats for different sub-tasks - like design, front-end, back-end (server and database), and we also had tasks there and git push notifications - so we could see in any time what someone is doing with application.
		
		Working on this project was really valuable experience that taught all of us how to better work as a team, how to concentrate on stuffs that we need to accomplish and not on someone of us as a one. We were many but we managed to work as a team and concentrate on on our objective - task we had to accomplish. That after all was and a good experience. We were really good team-mates and alongside developing application, we managed to have some fun and work in positive and constructive environment. We are satisfied with skills we acquired and are looking forward to work on some similar projects in same or other teams.
		
		\eject 