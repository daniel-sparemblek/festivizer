\chapter{Documentation Change Log}	
		\begin{longtabu} to \textwidth {|X[2, l]|X[13, l]|X[3, l]|X[3, l]|}
			\hline \multicolumn{1}{|l|}{\textbf{Rev.}}	& \multicolumn{1}{l|}{\textbf{Opis promjene/dodatka}} & \multicolumn{1}{|l|}{\textbf{Authors}} & \multicolumn{1}{l|}{\textbf{Date}} \\[3pt] \hline
			\endfirsthead
			
			\hline \multicolumn{1}{|l|}{\textbf{Rev.}}	& \multicolumn{1}{l|}{\textbf{Change/Addition Description}} & \multicolumn{1}{|l|}{\textbf{Authors}} & \multicolumn{1}{l|}{\textbf{Date}} \\[3pt] \hline
			\endhead
			
			\hline 
			\endlastfoot
			
			0.1 & Template uploaded to git.	& Bilic & 20.10.2019. 		\\[3pt] \hline 
			0.2 & Project Description written.	& Ceple & 6.11.2019. 		\\[3pt] \hline
			0.3 & Functional Requirements written.	& Ceple & 6.11.2019. 		\\[3pt] \hline
			0.4 & Class Diagram & Strbad & 12.11.2019. 		\\[3pt] \hline
			0.5 & Use Cases finished & Ceple & 13.11.2019. 		\\[3pt] \hline
			0.6 & Database description and diagrams added & Fribert & 13.11.2019. 		\\[3pt] \hline
			
			
		\end{longtabu}
	
		\textit{Moraju postojati glavne revizije dokumenata 1.0 i 2.0 na kraju prvog i drugog ciklusa. Između tih revizija mogu postojati manje revizije već prema tome kako se dokument bude nadopunjavao. Očekuje se da nakon svake značajnije promjene (dodatka, izmjene, uklanjanja dijelova teksta i popratnih grafičkih sadržaja) dokumenta se to zabilježi kao revizija. Npr., revizije unutar prvog ciklusa će imati oznake 0.1, 0.2, …, 0.9, 0.10, 0.11.. sve do konačne revizije prvog ciklusa 1.0. U drugom ciklusu se nastavlja s revizijama 1.1, 1.2, itd.}