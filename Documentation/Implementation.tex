\chapter{Implementation and User Interface}
		
		
		\section{Korištene tehnologije i alati}
		
			\textbf{\textit{dio 2. revizije}}
			
			 \textit{Detaljno navesti sve tehnologije i alate koji su primijenjeni pri izradi dokumentacije i aplikacije. Ukratko ih opisati, te navesti njihovo značenje i mjesto primjene. Za svaki navedeni alat i tehnologiju je potrebno \textbf{navesti internet poveznicu} gdje se mogu preuzeti ili više saznati o njima}.
			
			
			\eject 
		
		\section{Used technologies and tools}
		
		We have developed a mobile application for the Android operating system. The IDE of choice was Android Studio\footnote{\url{https://developer.android.com/}}. We have chosen it because it was developed by Google who are considered to be a sort of authority on Android. On the back-end we used Python\footnote{\url{https://www.python.org/}} hosted on the cloud at the site Pythonanywhere\footnote{\url{https://www.pythonanywhere.com/user/kaogrupa/}}.
		
		The database was also hosted on the aforementioned cloud. The database of choice was SQLite3\footnote{\url{https://www.sqlite.org/index.html}}. The pythonanywhere cloud offered the ability to start shell consoles and from there manage the database, as well as edit the Python back-end files. The back-end was made in Python Flask\footnote{\url{https://palletsprojects.com/p/flask/}} - a micro web framework. The cloud made the deployment and integration seamless and easy.
		
		Flask cooperates with further sub-components(as visible in the component diagram) - SQLAlchemy\footnote{\url{https://www.sqlalchemy.org/}}, Flask-JWT\footnote{\url{https://pythonhosted.org/Flask-JWT/}}, and Marshmallow\footnote{\url{https://marshmallow.readthedocs.io/en/stable/}}. SqlAlchemy is the component used for communicating with the database, JWT is used for the OAuth 2.0 implementation\footnote{\url{https://oauth.net/2/}}, and Marshmallow is used for JSON serialisation and de-serialisation. JWT = JSON Web Token.
		
		Front-end design and logic were all developed in Android Studio which provided a very convenient IDE --> code-generation and assistance was of great help, as well as the ability to edit XML files through a GUI instead of only the classical text editing. The application can be deployed and/or debugged on the mobile device by connecting it to the PC/Mac that has Android Studio running with the application code loaded into it. This has provided the basis for smooth application development and testing.
		
		Team communication was at first achieved using the Whatsapp application.\footnote{\url{https://www.whatsapp.com//}}. This worked at first, but has since proven itself to not provide satisfactory level of organisation and communication features. Therefore it was replaced with Slack\footnote{\url{https://slack.com/}} - where multiple threads and channels have made organisation and project overview much easier. What's more, a Slack bot and tasks have further increased the productivity and organisation.
		
		The version control system used was Git\footnote{\url{https://git-scm.com/}}. The most important branches were: master, dev, and devdoc. The application code was put on the dev branch, while the documentation resided on the devdoc branch. Before turning the application in, branched would be merged to the main branch --> the master branch. 
		
		The software used for documentation were: LaTeX\footnote{\url{https://www.latex-project.org/}} and Astah UML\footnote{\url{http://astah.net/}}. For LaTeX, we were provided with a template which greatly helped with LaTeX understanding and documentation writing. Astah UML was used for drawing diagrams and ha proven to be very accessible, intuitive, and satisfying to use.
		
		\eject 
	
		\section{Ispitivanje programskog rješenja}
			
			\textbf{\textit{dio 2. revizije}}\\
			
			 \textit{U ovom poglavlju je potrebno opisati provedbu ispitivanja implementiranih funkcionalnosti na razini komponenti i na razini cijelog sustava s prikazom odabranih ispitnih slučajeva. Studenti trebaju ispitati temeljnu funkcionalnost i rubne uvjete.}
	
			
			\subsection{Ispitivanje komponenti}
			\textit{Potrebno je provesti ispitivanje jedinica (engl. unit testing) nad razredima koji implementiraju temeljne funkcionalnosti. Razraditi \textbf{minimalno 6 ispitnih slučajeva} u kojima će se ispitati redovni slučajevi, rubni uvjeti te izazivanje pogreške (engl. exception throwing). Poželjno je stvoriti i ispitni slučaj koji koristi funkcionalnosti koje nisu implementirane. Potrebno je priložiti izvorni kôd svih ispitnih slučajeva te prikaz rezultata izvođenja ispita u razvojnom okruženju (prolaz/pad ispita). }
			
			
			
			\subsection{Ispitivanje sustava}
			
			 \textit{Potrebno je provesti i opisati ispitivanje sustava koristeći radni okvir Selenium\footnote{\url{https://www.seleniumhq.org/}}. Razraditi \textbf{minimalno 4 ispitna slučaja} u kojima će se ispitati redovni slučajevi, rubni uvjeti te poziv funkcionalnosti koja nije implementirana/izaziva pogrešku kako bi se vidjelo na koji način sustav reagira kada nešto nije u potpunosti ostvareno. Ispitni slučaj se treba sastojati od ulaza (npr. korisničko ime i lozinka), očekivanog izlaza ili rezultata, koraka ispitivanja i dobivenog izlaza ili rezultata.\\ }
			 
			 \textit{Izradu ispitnih slučajeva pomoću radnog okvira Selenium moguće je provesti pomoću jednog od sljedeća dva alata:}
			 \begin{itemize}
			 	\item \textit{dodatak za preglednik \textbf{Selenium IDE} - snimanje korisnikovih akcija radi automatskog ponavljanja ispita	}
			 	\item \textit{\textbf{Selenium WebDriver} - podrška za pisanje ispita u jezicima Java, C\#, PHP koristeći posebno programsko sučelje.}
			 \end{itemize}
		 	\textit{Detalji o korištenju alata Selenium bit će prikazani na posebnom predavanju tijekom semestra.}
			
			\eject 
		
		
		\section{Dijagram razmještaja}
			
			\textbf{\textit{dio 2. revizije}}
			
			 \textit{Potrebno je umetnuti \textbf{specifikacijski} dijagram razmještaja i opisati ga. Moguće je umjesto specifikacijskog dijagrama razmještaja umetnuti dijagram razmještaja instanci, pod uvjetom da taj dijagram bolje opisuje neki važniji dio sustava.}
			
			\eject 
			
		\section{UML Deployment Diagram}
		
			Due to the application including a lot of users, we have opted for the specification deployment diagram.
			The application consists of two parts: the mobile application(kind of a front-end) and the back-end hosted on the Pythonanywhere cloud. The application uses protocols GET(for information retrieval from the server), and POST in order to send/communicate information to the server. Furthermore, the server consists of 4 parts:
			\begin{itemize}
				\item App.db - definition of the database for SQLite 3
				\item Models.py - database tables modelled as Python classes
				\item Run.py - endpoint initialisation and control
				\item Resource.py - Used for handling server requests and responses
			\end{itemize}
			
			As far as hardware goes - basically 2 devices are required - the mobile phone and the cloud hosting computer. Network is also required as it is used to convey
			information between these 2 endpoints. While there is only one server instance, it is expected that multiple concurrent mobile phones will be using the application. Therefore, in the instance deployment diagram more than one mobile phone could be present.
		
			\eject
		
		\section{Deployment instructions}
		
			\subsection{Setting up the server side}
				
				The files are already present on the URI: https://www.pythonanywhere.com/user/kaogrupa/. It is merely necessary to run the server using the button depicted down below:
				####################		IMAGE HERE			###########################
				
				The server, once run, initialises and sets up the database, and begins listening for requests. 
			
			\textbf{\textit{dio 2. revizije}}\\
		
			 \textit{U ovom poglavlju potrebno je dati upute za puštanje u pogon (engl. deployment) ostvarene aplikacije. Na primjer, za web aplikacije, opisati postupak kojim se od izvornog kôda dolazi do potpuno postavljene baze podataka i poslužitelja koji odgovara na upite korisnika. Za mobilnu aplikaciju, postupak kojim se aplikacija izgradi, te postavi na neku od trgovina. Za stolnu (engl. desktop) aplikaciju, postupak kojim se aplikacija instalira na računalo. Ukoliko mobilne i stolne aplikacije komuniciraju s poslužiteljem i/ili bazom podataka, opisati i postupak njihovog postavljanja. Pri izradi uputa preporučuje se \textbf{naglasiti korake instalacije uporabom natuknica} te koristiti što je više moguće \textbf{slike ekrana} (engl. screenshots) kako bi upute bile jasne i jednostavne za slijediti.}
			
			
			 \textit{Dovršenu aplikaciju potrebno je pokrenuti na javno dostupnom poslužitelju. Studentima se preporuča korištenje neke od sljedećih besplatnih usluga: \href{https://aws.amazon.com/}{Amazon AWS}, \href{https://azure.microsoft.com/en-us/}{Microsoft Azure} ili \href{https://www.heroku.com/}{Heroku}. Mobilne aplikacije trebaju biti objavljene na F-Droid, Google Play ili Amazon App trgovini.}
			
			
			\eject 